\documentclass[noserifmath,aspectratio=169]{beamer}

\usetheme{leipzig}

\usepackage{booktabs}

\newcommand{\themename}{\textbf{leipzig}~}

\title[Leipzig]{Leipzig}
\subtitle[Beamer Theme]{Beamer theme}
\author{Author}
\institute{Institute}
\date{\today}
\titlegraphic{}
\location{Leipzig}
\thankstitle{Thank You!}
\address{Beethovenstraße 15, 04107 Leipzig}
\email{example@uni-leipzig.de}
\phone{+49 1234 5678910}
%\website{}

\begin{document}

\maketitle

\begin{frame}{Table of contents}
  \setbeamertemplate{section in toc}[sections numbered]
  \tableofcontents[hideallsubsections]
\end{frame}

\section{Introduction}

\begin{frame}[fragile]{Leipzig}
The \themename theme is a Beamer theme that complies with the corporate design of the Leipzig University.

Enable the theme by loading
\begin{verbatim}
    \documentclass{beamer}
    \usetheme{leipzig}
\end{verbatim}

Note that this package only works with XeTeX/XeLaTeX.
\end{frame}

\section{Elements}
\begin{frame}[fragile]{Package Options}
\begin{description}
\item[\texttt{noserifmath}] displays math using Arial
\item[\texttt{nosectionpages}] hides section pages.
\end{description}
The theme is suitable for 16:9 as well as 4:3 aspect ratios.
\end{frame}

\begin{frame}[fragile]{Title Page Macros}
The usual title page macros such as author, date, title and subtitle are supported.
Additional options include:
\begin{itemize}
\item short title for use in the header: \verb+\title[short title]{long title}+ (analogous for subtitle)
\item institution for use in the footer: \\ \verb+\institute{institution}+
\item additional graphics on the title page: \\ \verb+\titlegraphic{filename}+
\item location of the talk: \verb+\location{location}+
\end{itemize}
\end{frame}


\begin{frame}[fragile]{Thanks Page Macros}
A ``Thank You!''-Page can be displayed using \verb+\makethanks+. The displayed information can be specified using:
\begin{itemize}
\item \verb+\thankstitle{}+ for a customized thank-you text, default is ``Vielen Dank''
\item \verb+\address{}+ 
\item \verb+\email{}+
\item \verb+\phone{}+
\item \verb+\website{}+ - default is \texttt{www.uni-leipzig.de}, leave empty to disable
\end{itemize}
\end{frame}

\begin{frame}[fragile]{Typography}
\begin{verbatim}
You can \emph{emphasize} text, \alert{accent} 
parts or show \textbf{bold} results.
\end{verbatim}
\begin{center}becomes\end{center}
You can \emph{emphasize} text, \alert{accent} parts or show \textbf{bold} results.\footnote{Footnotes work as well.}
\end{frame}

\begin{frame}{Font features}
  \begin{itemize}
    \item Regular
    \item \textit{Italic}
    \item \textbf{Bold}
    \item \textbf{\textit{Bold Italic}}
    \item \texttt{Monospace}
    \item \texttt{\textit{Monospace Italic}}
    \item \texttt{\textbf{Monospace Bold}}
    \item \texttt{\textbf{\textit{Monospace Bold Italic}}}
  \end{itemize}
\end{frame}

\begin{frame}{Lists}
  \begin{columns}[T,onlytextwidth]
    \column{0.33\textwidth}
      Items
      \begin{itemize}
        \item Item \item Item \item Item
      \end{itemize}

    \column{0.33\textwidth}
      Enumerations
      \begin{enumerate}
        \item First \item Second  \item Last
      \end{enumerate}

    \column{0.33\textwidth}
      Descriptions
      \begin{description}
        \item[Caption] Text 
        \item[Caption] Text
        \item[Caption] Text
      \end{description}
  \end{columns}
\end{frame}

\begin{frame}{Figures}
  \begin{figure}
    \begin{tikzpicture}[scale=1]
    \colorlet{col1}{alerted text.fg!80}
    \colorlet{col2}{alerted text.fg!60}
    \colorlet{col3}{alerted text.fg!40}
    \colorlet{col4}{alerted text.fg!20}
   	\draw [help lines] (-4.25,-1.25) grid (4.25,1.5);
   	\draw [help lines,step=0.25cm] (-2.99,0) grid (2.99,0.99);

   \begin{scope}[smooth,draw=gray!20,y=0.3989422804cm]
        \filldraw [fill=col3] plot[id=f1,domain=-3:-2] function {exp(-x*x/2)}
            -- (-2,0) -- (-3,0) -- cycle;
        \filldraw [fill=col2] plot[id=f2,domain=-2:-1] function {exp(-x*x/2)}
            -- (-1,0) -- (-2,0) -- cycle;
        \filldraw [fill=col1] plot[id=f3,domain=-1:0]  function {exp(-x*x/2)}
            -- (0,0)  -- (-1,0) -- cycle;
        \filldraw [fill=col1] plot[id=f4,domain=0:1] function {exp(-x*x/2)}
            -- (1,0)  --  (0,0) -- cycle;
        \filldraw [fill=col2] plot[id=f5,domain=1:2] function {exp(-x*x/2)}
            -- (2,0)  -- (1,0) -- cycle;
        \filldraw [fill=col3] plot[id=f6,domain=2:3] function {exp(-x*x/2)}
            -- (3,0)  -- (2,0) -- cycle;
        \draw[black] plot[id=f7,domain=-4.25:4.25,samples=100]
            function {exp(-x*x/2)};
   \end{scope}
       \draw[->] (-4.25,0) -- (4.25,0) node [right] {$x$};

    \foreach \pos/\label in {-3/$-3\sigma$,-2/$-2\sigma$,-1/$-\sigma$,
            1/$\sigma$,2/$2\sigma$,3/$3\sigma$}
        \draw (\pos,0) -- (\pos,-0.1) (\pos cm,-3ex) node
            [anchor=base,fill=white,inner sep=1pt]  {\label};

    \draw (-0.1,1) -- (.1,1) node [right,fill=white,inner sep=1pt] {$\sigma$};

    \foreach \pos/\percent/\height in {1/34/0.5,2/14/0.25,3/2/0.125,4/0.1/0.1}
    {
      \node[text=col\pos,anchor=base,yshift=2pt,xshift=-0.625cm,
        inner sep=1pt] at (\pos,\height) {$\percent\%$};
      \node[text=col\pos,anchor=base,yshift=2pt,xshift=.625cm,
        inner sep=1pt]  at (-\pos,\height) {$\percent\%$};
    }
\end{tikzpicture}
    \caption{Standard deviation from \href{http://texample.net/tikz/examples/standard-deviation/}{texample.net}.}
  \end{figure}
\end{frame}

\begin{frame}{Tables}
  \begin{table}
    \caption{Largest cities in the world (source: Wikipedia)}
    \begin{tabular}{@{} lr @{}}
      \toprule
      City & Population\\
      \midrule
      Mexico City & 20,116,842\\
      Shanghai & 19,210,000\\
      Peking & 15,796,450\\
      Istanbul & 14,160,467\\
      \bottomrule
    \end{tabular}
  \end{table}
\end{frame}

\begin{frame}{Blocks}
  	Three different block environments are pre-defined.
    \begin{block}{Default}
        Block content.
    \end{block}

      \begin{alertblock}{Alert}
        Block content.
      \end{alertblock}

      \begin{exampleblock}{Example}
        Block content.
      \end{exampleblock}
\end{frame}

\begin{frame}{Math}
  \begin{equation*}
    \frac{1}{\sigma\sqrt{2\pi}}\exp\biggl(\frac{-x^2}{2\sigma^2}\biggr)
  \end{equation*}
\end{frame}

\begin{frame}[fragile]{Citations}
For a general introduction to creating slides with Beamer, refer to the Beamer manual.\cite{BeamerManual}
To split your bibliography into multiple frames, use the \verb+[allowframebreaks]+ option.
\end{frame}

\makethanks

\section{References}
\begin{frame}[allowframebreaks]
    \bibliographystyle{plain}
	\bibliography{bibliography.bib}
\end{frame}

\section{Appendix}
\begin{frame}{Appendix}
    Things that differ in the \LaTeX~ version compared to the original \texttt{.pptx} style:
        \begin{itemize}
            \item the footer does not include the university wordmark, as a vector-based graphic file of such is not obtainable from the university
            \item the frame titles are not converted to uppercase, as this introduces a bug with the \texttt{[allowframebreaks]} option
        \end{itemize}
\end{frame}

\end{document}
